\section{结论与展望}
\subsection{结论}
智能网联汽车由于其大量集成了各类先进技术,复杂性相较于传统汽车有很大的提高。也正因如此,智能网联汽车比起传统汽车有可能会面临到更多的安全风险。本文关注智能网联汽车在GNSS位置攻击欺骗方面的检测算法工作,并在实验中成功验证了所设计的检测算法的有效性。更进一步地,本文还基于ISO 26262标准,对智能网联汽车在受到GNSS位置欺骗攻击后可能面临的功能安全风险进行了分析,并提出了相应的应急策略,从而实现检测算法与应急策略的联动。
\subsection{进一步研究工作}
尽管本文在实验中验证了所设计的检测算法可以有效完成检测工作,但是并没有将不同场景下GNSS定位的精确度纳入考虑。具体而言,欺骗阈值$\gamma$应该在不同的场景下,如城市高楼间、平原、隧道内、地下停车场等,具有一定不同的值。但由于目前缺少各种场景中GNSS定位精确度的相关数据,这项工作暂时无法进行。下一步将考虑在实车上采集数据,从而完善检测算法在不同场景下的表现。
