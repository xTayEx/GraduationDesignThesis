\section{GNSS位置欺骗攻击功能安全应急策略}
\subsection{智能网联汽车GNSS功能与功能安全风险项分析}
要设计功能安全应急策略,首先要分析可能会遇到安全风险的功能有哪些。在智能网联汽车中,GNSS模块主要负责为车辆自动驾驶算法模块提供当前位置、速度、转向角,从而保证自动驾驶算法可以及时规划路线。当该功能因受到GNSS位置欺骗攻击而出现故障时,有可能会出现不同的功能安全风险。具体可见表\ref{tab:gnss_shixiao}。
\begin{table}[]
    \begin{center}
    \begin{tabular}{|c|c|}
    \hline
    GNSS模块功能 & \multicolumn{1}{c|}{功能安全风险}\\ \hline
    \multirow{2}{*}{\begin{tabular}[c]{@{}c@{}}为车辆自动驾驶算法\\ 提供当前位置、速度、转向角\end{tabular}} & \begin{tabular}[c]{@{}c@{}}车辆远离正确规划路线,\\ 但车辆目前所处环境较为安全\\ (周围无障碍物、行人等)\end{tabular}     \\ \cline{2-2}
 & \begin{tabular}[c]{@{}c@{}}车辆远离正确规划路线,\\ 且车辆目前所处环境复杂\\ (山路、窄路、雨天、湖泊或河流旁等)\end{tabular} \\ \hline
    \end{tabular}
    \end{center}
    \caption{GNSS模块在受到欺骗后可能遇到的功能安全风险}
    \label{tab:gnss_shixiao}
\end{table}
\subsection{应用HARA方法划分风险等级}
\paragraph{车辆远离正确规划路线,但车辆所处环境较为安全}
对于该风险,由于车辆所处的环境比较安全,周围没有明显障碍物或行人,因此,发生交通事故的可能性会相对较低,严重度可以认为是S1。对于暴露度,由于当前智能网联汽车多属于消费级汽车,所处环境一般位于城市道路、小区道路等,处于空旷环境的概率较小,因此E因子应被划分为E2;最后,对于C因子,由于此时所处环境较为简单空旷,因此驾驶人一般可以有足够的时间接管汽车并调整方向,故C因子应该被划分为C1。对照表\ref{tab:ASIL},该功能安全风险的等级为QM。
\paragraph{车辆远离正确规划路线,且车辆所处环境复杂}
对于该风险,由于车辆所处环境比较复杂,周围有明显障碍物或行人,因此,相比在安全环境中,此时会更容易发生交通事故,故严重度应该被分类为S3。如上文所述,当前智能网联汽车多处于城市道路等复杂环境,但也会有处于空旷地区等简单环境的情况。因此,可以认为E因子为E3。最后,由于此时所处环境复杂,一旦发生功能失效,司机往往较难在事故发生前接管并控制车辆脱离风险,因此,认为C因子为C3。综上,对照表\ref{tab:ASIL},该功能安全风险的等级为C。

上文对具体功能所涉及到的S、E、C因子以及ASIL进行了划分。具体见表\ref{tab:gnss_sec_asil_zongjie}。
\begin{table}
\begin{center}
    \begin{tabular}{ccccc}
    \hline
    功能安全风险                                                            & 严重度 & 暴露度 & 可控性 & ASIL \\ \hline
    \begin{tabular}[c]{@{}c@{}}车辆远离规划路线,\\ 但车辆目前所处环境较为安全\end{tabular} & S1  & E2  & C1  & QM   \\ \hline
    \begin{tabular}[c]{@{}c@{}}车辆远离规划路线,\\ 且车辆目前所处环境复杂\end{tabular}   & S3  & E3  & C3  & C    \\ \hline
    \end{tabular}
\end{center}
\caption{对GNSS模块受到位置欺骗攻击后可能会遇到的功能安全风险使用HARA方法分析,得到的S、E、C因子以及ASIL}
\label{tab:gnss_sec_asil_zongjie}
\end{table}

\subsection{设计应急策略}
对于上述两种功能安全风险,本文分别提出以下功能安全应急策略:
\paragraph{车辆远离规划路线,但车辆目前所处环境较为安全}
该风险的ASIL为QM。若车辆面临该风险,则尽可能调低GNSS模块在汽车定位算法中的权重,同时提高其他模块,如IMU、LiDAR、RADAR的权重,GNSS模块同时应根据其他传感器的定位结果恢复正常功能;另外,车辆进入辅助驾驶模式,由驾驶人接管车辆驾驶,自动驾驶模块仅为驾驶人提供有限的、必要的辅助驾驶功能(如制动辅助)。
\paragraph{车辆原理规划路线,且车辆目前所处环境复杂}
该风险的ASIL为C。若车辆面临该风险,则自动驾驶模块应立即将驾驶权交由驾驶人接管。同时,GNSS模块应借助IMU、LiDAR、RARAR等传感器得出的定位结果尽可能恢复正常功能。
\subsection{本章小结}
本章主要对智能网联汽车在GNSS模块受到位置欺骗攻击的情况下有可能面临的功能安全风险进行了分析,并使用HARA方法划分了对应的风险等级。同时,在风险等级的基础上,针对两种功能安全风险,提出了对应的应急策略。
