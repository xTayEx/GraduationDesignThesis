\newpage


\centerline{\fangsong\bf\zihao{-2}{智能网联汽车GNSS位置欺骗攻击}}
\vspace{\baselineskip}
\centerline{\fangsong\bf\zihao{-2}{与功能安全危害联动预警策略设计及实现}}
\addcontentsline{toc}{section}{摘要(关键词)}%加入目录


\vskip 1cm

\begin{center}
	\kaishu
	\hspace{2cm}计算机与软件学院计算机科学与技术专业 \quad 李宇良
	\vspace{5bp}
	\newline
	学号:2018151004
\end{center}

\vskip 10bp

{
\kaishu
\hspace{5bp}{\zihao{-4}\textbf{【摘要】}}
随着机器学习、数据挖掘等技术的蓬勃发展,各种新兴技术也在不断产生。这其中就包括智能网联汽车。智能网联汽车区别于传统汽车,其搭载了各类传感器、执行器,同时结合了现代通信互联技术,可以实现车与车、车与人、车与路面等信息交换。然而,与智能网联汽车相关的GNSS位置欺骗问题也频频困扰各大厂商与研究者。因此,本文主要研究智能网联汽车上的GNSS位置欺骗攻击检测,以及相应的功能安全应急策略。本文的主要工作可以可概括为以下三点:
\begin{enumerate}
    \item 针对智能网联汽车上的GNSS位置欺骗攻击,基于机器学习中的LSTM网络,使用公开数据集建立训练集,设计并实现了一个可以有效实现攻击检测的算法。训练LSTM网络需要的数据为速度、转向角、向前加速度以及相邻时间戳之间车辆的移动距离。本文通过使用哈弗森大圆公式,将原始数据中的GNSS定位点数据转化为了距离。
    \item 在公开数据集上做一定的更改,得到欺骗数据集,并以此作为测试集进行了实验。实验结果表明,本文设计的算法可以有效检测到针对智能网联汽车的GNSS位置欺骗攻击。
    \item 依据ISO 26262标准,使用HARA方法对智能网联汽车在遭受GNSS位置欺骗攻击的情况下可能面临的功能安全风险进行了分析,同时划分了相应的风险等级,并设计了合适的应急策略。从而实现算法检测与应急策略的联动。
\end{enumerate}
综上所述,本文基于LSTM技术与ISO 26262标准,针对GNSS模块设计了位置欺骗攻击的检测算法,同时设计了对应的功能安全应急策略。检测算法可以有效工作,但在场景细化方面仍有提升空间。后续可以通过实车采集数据获得不同场景下的GNSS定位误差,从而细化不同驾驶场景中的检测表现。

\vskip 10bp

\hspace{5bp} {\zihao{-4}\textbf{【 关键词】}}
智能网联汽车;GNSS;功能安全;LSTM
}
