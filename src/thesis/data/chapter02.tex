\section{相关技术简介}

\subsection{GNSS(全球卫星导航系统)概述}
全球卫星导航系统(Global Navigation Satellite System,下称GNSS),一般是指通过覆盖全球的导航卫星系统为地面或近地面用户提供全天候的三维空间坐标以及时间信息的无线定位系统。使用GNSS进行定位的用户可以通过具有GNSS信号接收器接收来自当前区域卫星的定位信号,并通过一系列的解码与计算得到较为准确的空间信息与时间信息,从而实现定位、导航、授时(PNT)的功能。

世界上第一个全球卫星导航系统是美国的GPS系统。该系统在设计之处一共由24颗卫星组成,其中21颗为工作卫星,3颗为备用卫星。而截至到目前,GPS系统的卫星数目已经达到了31颗。而在我国,第一颗北斗卫星在2007年4月14日发射,被在往后的若干年里不断完善北斗导航系统。截至2020年,北斗系统已经实现向全球提供服务的目标,与美国GPS、俄罗斯GLONASS、欧盟GALILEO并列成为四大全球定位系统。除了上述的全球性定位系统外,还包括区域系统和增强系统。其中区域系统有日本的QZSS和印度的IRNSS;增强系统则包括美国的WASS、日本的MSAS以及欧盟的EGNOS等。

GNSS的定位原理可以认为是求解一组方程。对于用户所处空间位置$(x_u, y_u, z_u)$,由于导航卫星所处的精确位置是可知的,同时卫星与用户之间的距离也可通过光速与时间差得到,因此可以列出以下方程组。
\begin{equation}
    \begin{cases}
        \rho_1=\sqrt{(x_1-x_u)^2+(y_1-y_u)^2+(z_1-z_u)^2}\\
        \rho_2=\sqrt{(x_2-x_u)^2+(y_2-y_u)^2+(z_2-z_u)^2}\\
        \rho_3=\sqrt{(x_3-x_u)^2+(y_3-y_u)^2+(z_3-z_u)^2}\\
    \end{cases}
    \label{eq:1}
\end{equation}
其中,$x_i$,$y_i$,$z_i$表示当前用于定位用户位置的第$i$颗卫星的空间位置。$\rho_i$则表示用户距离第$i$颗卫星的距离。求解上述方程组,即可求得用户位置坐标$(x_u,y_u,z_u)$。然而,在实际应用中,除了上述的三个未知数以外,往往还需要第四个未知数$t_u$作为修正项。原因在于,在计算用户位置与卫星间距离时,需要使用导航卫星中的原子钟与地面用户接收器的时钟作差得到钟差,但接收器的时钟精度要比原子钟精度低。这就导致最终得到的钟差会有一定的误差,因此需要加入修正项。此时的方程组为。
\begin{equation}
    \begin{cases}
        \rho_1=\sqrt{(x_1-x_u)^2+(y_1-y_u)^2+(z_1-z_u)^2}+ct_u\\
        \rho_2=\sqrt{(x_2-x_u)^2+(y_2-y_u)^2+(z_2-z_u)^2}+ct_u\\
        \rho_3=\sqrt{(x_3-x_u)^2+(y_3-y_u)^2+(z_3-z_u)^2}+ct_u\\
        \rho_4=\sqrt{(x_4-x_u)^2+(y_4-y_u)^2+(z_4-z_u)^2}+ct_u\\
    \end{cases}
    \label{eq:2}
\end{equation}
其中,$c$表示光速。
\subsection{GNSS位置欺骗攻击及检测方法概述}
目前,GNSS位置欺骗攻击还没有精确的定义。一般而言,“欺骗”是指某人或某程序利用数据篡改、数据伪造等手段成功伪装成另一个人或另一个程序,其目的往往是获取情报或影响被攻击者的正常运作。具体到GNSS位置欺骗攻击方面,攻击者会通过伪造错误定位信号或转发真实卫星信号等手段进行攻击。遭受欺骗攻击的GNSS接收器则会计算出一个错误的位置或错误的时间,从而导致依赖于GNSS定位的其他部件工作受阻或出错,甚至是无法工作。下面将对常见的GNSS位置欺骗攻击手段及检测方法进行简要介绍。
\subsubsection{欺骗方法概述}
\paragraph{基于信号模拟器的自主产生式攻击}

该攻击方式的主要思想是通过一个GNSS信号模拟器发送虚假GNSS定位信号来实现欺骗目的。目前,诸如Spirent公司的GSS8000等GNSS信号模拟器可以模拟出各种真实环境中的卫星定位信号。通过向接收机发送生成的定位信号的方法,可以实现一定程度上的位置欺骗攻击。但这种方法的缺点也很明显。由于GNSS信号模拟器所产生的信号是完全自主产生的,并没有与实际卫星进行信号同步,所以很容易导致接收机出现失锁或重捕的问题,从而导致欺骗被检测。另外,GNSS模拟器庞大的体积以及高昂的价格也是该欺骗方法的主要缺点之一。
\paragraph{基于接收机的接受产生式攻击}

该攻击方式所使用的干扰源主要由两部分组成,即接收机与信号模拟器。接受产生式攻击的基本工作原理与自主产生式攻击类似,都需要使用一个信号模拟器进行信号模拟生产。两者最大的不同在于,前者在产生虚假定位信号时所使用的参数由操作者或机器自身自主设置;而后者则是根据接收机接收到的真是卫星信号的估计结果,通过算法计算得到。与自主产生式攻击相比,接收产生式攻击所产生的信号与真实信号接近,其隐蔽性要更强。而其缺点在于,由于在使用真实卫星信号计算模拟参数时需要精确测定目标接收机与欺骗干扰源之间的三维位置关系,实现难度较大。尤其是当欺骗目标处于运动状态的时候,需要实时测定两者位置关系。也正因如此,这种欺骗手段一般只用于静止状态或低速运动状态下的目标。
\paragraph{基于信号转发器的转发式攻击}

我国在GNSS欺骗方面的研究主要集中在转发式攻击。该攻击手段所采用的思路与上述两种欺骗方法截然不同。该方法不再生成虚假信号,而是直接使用真实的卫星导航信号,通过使用转发器发射到目标区域,从而使目标接收到另一个空间位置的GNSS定位信号。与产生式攻击方法相比,该方法最大的优点在于不需要了解信号的内部细节(如GNSS信号格式、加密方式等),从而避免了大量技术细节与限制,并因此扩大了适用范围。而该方法的最大缺点在于,由于在信号转发时需要对原始信号放大,这会导致信号中的噪声被一起放大,从而导致转发信号与接收机接收到的真实定位信号在噪声水平上有较大的差别,容易被检测到。
\subsubsection{检测方法概述}
\paragraph{基于空间信息处理的检测方法}

该检测方法的主要思想是通过判别导航信号的空间信息来实现欺骗检测。具体来说,欺骗源的欺骗信号往往来自于
\paragraph{基于信号到达时间的检测方法}
\paragraph{基于机器学习的检测方法}


\subsection{LSTM概述}
\subsection{汽车功能安全概述}
\subsection{国内外研究现状}
\subsubsection{智能网联汽车GNSS位置欺骗攻击检测方面}
\subsubsection{基于预测的GNSS位置欺骗攻击检测算法}
\subsubsection{智能网联汽车GNSS功能安全方面}
\subsection{本章小结}
