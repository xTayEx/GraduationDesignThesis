\section{相关技术简介}

\subsection{GNSS(全球卫星导航系统)概述}
全球卫星导航系统(Global Navigation Satellite System,下称GNSS),一般是指通过覆盖全球的导航卫星系统为地面或近地面用户提供全天候的三维空间坐标以及时间信息的无线定位系统。使用GNSS进行定位的用户可以通过具有GNSS信号接收器接收来自当前区域卫星的定位信号,并通过一系列的解码与计算得到较为准确的空间信息与时间信息,从而实现定位、导航、授时(PNT)的功能。

世界上第一个全球卫星导航系统是美国的GPS系统。该系统在设计之处一共由24颗卫星组成,其中21颗为工作卫星,3颗为备用卫星。而截至到目前,GPS系统的卫星数目已经达到了31颗。而在我国,第一颗北斗卫星在2007年4月14日发射,被在往后的若干年里不断完善北斗导航系统。截至2020年,北斗系统已经实现向全球提供服务的目标,与美国GPS、俄罗斯GLONASS、欧盟GALILEO并列成为四大全球定位系统。除了上述的全球性定位系统外,还包括区域系统和增强系统。其中区域系统有日本的QZSS和印度的IRNSS;增强系统则包括美国的WASS、日本的MSAS以及欧盟的EGNOS等。

GNSS的定位原理可以认为是求解一组方程。对于用户所处空间位置$(x_u, y_u, z_u)$,由于导航卫星所处的精确位置是可知的,同时卫星与用户之间的距离也可通过光速与时间差得到,因此可以列出以下方程组。
\begin{equation}
    \begin{cases}
        \rho_1=\sqrt{(x_1-x_u)^2+(y_1-y_u)^2+(z_1-z_u)^2}\\
        \rho_2=\sqrt{(x_2-x_u)^2+(y_2-y_u)^2+(z_2-z_u)^2}\\
        \rho_3=\sqrt{(x_3-x_u)^2+(y_3-y_u)^2+(z_3-z_u)^2}\\
    \end{cases}
    \label{eq:1}
\end{equation}
其中,$x_i$,$y_i$,$z_i$表示当前用于定位用户位置的第$i$颗卫星的空间位置。$\rho_i$则表示用户距离第$i$颗卫星的距离。求解上述方程组,即可求得用户位置坐标$(x_u,y_u,z_u)$。然而,在实际应用中,除了上述的三个未知数以外,往往还需要第四个未知数$t_u$作为修正项。原因在于,在计算用户位置与卫星间距离时,需要使用导航卫星中的原子钟与地面用户接收器的时钟作差得到钟差,但接收器的时钟精度要比原子钟精度低。这就导致最终得到的钟差会有一定的误差,因此需要加入修正项。此时的方程组为。
\begin{equation}
    \begin{cases}
        \rho_1=\sqrt{(x_1-x_u)^2+(y_1-y_u)^2+(z_1-z_u)^2}+ct_u\\
        \rho_2=\sqrt{(x_2-x_u)^2+(y_2-y_u)^2+(z_2-z_u)^2}+ct_u\\
        \rho_3=\sqrt{(x_3-x_u)^2+(y_3-y_u)^2+(z_3-z_u)^2}+ct_u\\
        \rho_4=\sqrt{(x_4-x_u)^2+(y_4-y_u)^2+(z_4-z_u)^2}+ct_u\\
    \end{cases}
    \label{eq:2}
\end{equation}
\subsection{GNSS位置欺骗攻击及检测方法概述}
\subsubsection{欺骗方法概述}
\subsubsection{检测方法概述}

\subsection{LSTM概述}
\subsection{汽车功能安全概述}
\subsection{国内外研究现状}
\subsubsection{智能网联汽车GNSS位置欺骗攻击检测方面}
\subsubsection{基于预测的GNSS位置欺骗攻击检测算法}
\subsubsection{智能网联汽车GNSS功能安全方面}
\subsection{本章小结}
