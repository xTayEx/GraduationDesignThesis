\section{引言}

\subsection{研究背景及意义}
近年来,随着现代通信技术以及自动驾驶技术的迅速发展,汽车这一传统出行载体也在往智能化、互联化的发展方向迈进。由此诞生出来的新产物便是智能网联汽车。区别于一般的自动驾驶汽车(ADAS),智能网联汽车可以理解为在自动驾驶技术的基础上(即自动驾驶决策单元、对应的传感器、控制器等),将车联网技术融合其中,使得汽车可以与周围环境、道路、甚至“云”,进行信息的沟通与共享,从而实现V2X(Vehicle to X)[智能网联汽车信息安全威胁识别和防护方法研究\_郝晶晶]。这种互联化的技术可以使得传统自动驾驶汽车拥有更全面复杂的环境感知能力与决策能力,从而提高自动驾驶汽车的安全性与可靠性,并最终实现可以替代驾驶员所有操作的“无人驾驶汽车”。

外部网络接入带来的不仅有自动驾驶汽车各项能力的提升,随之而来的还有针对智能网联汽车的信息安全威胁。据国家工信部统计,自2020年以来,针对车联网信息服务提供商、整车企业等相关企业的恶意攻击高达280万起[引用白皮书];另外,截止到2020年底,全球范围内共发现110个与汽车产品相关的CVE漏洞。这些漏洞涉及范围广泛,包含汽车的内部网络、网关、传感器、车载信息娱乐系统、蓝牙、OBD端口等等部件。这些针对汽车产品的安全漏洞以及攻击不仅会影响用户的信息娱乐服务质量,威胁用户的信息安全,甚至还很有可能导致汽车控制功能失效,直接威胁车内乘客的人生安全。由此可见,智能网联汽车相关的信息安全问题亟待解决。

一般而言,与智能网联汽车相关联的信息风险可以分为IP流量攻击风险,CAN流量攻击风险,GNSS位置欺骗攻击,蓝牙攻击风险以及车机攻击风险[这里引用一个文章]。本文主要关注GNSS位置欺骗攻击,其中包括攻击检测以及对应功能安全危害的预警策略。

GNSS位置欺骗攻击最早出现在军事领域。2011年12月,伊朗使用GNSS位置欺骗攻击技术,成功控制了美军的RQ-170“哨兵”无人机,使其降落到伊朗机场。2016年1月,美国海军的两艘小型巡逻艇在执行任务时偏离原本的航行路线,进入了伊朗海域,从而使船只与美国军方失去联系。而在民用领域,2014年3月,从吉隆坡国际机场飞往北京首都机场的MH370航班在航行过程中失联,迄今尚未发现任何残骸。一些专家认为,MH370很可能受到欺骗性的干扰,导致其偏离航线并在耗尽燃料后坠毁。从技术角度来看,GNSS位置欺骗攻击确实具有这种潜在的攻击力。[S. Bian, Y. Hu, and B. Ji, ‘‘Research status and prospect of GNSS anti-spoofing technology,’’ Scientia Sinica Informationis, vol. 47, no. 3, pp. 275–287, 2017.]近十年来,随着对该类攻击的深入研究,学术界已经有多种相对成熟的攻击检测方法。然而,这些研究大多数集中在军事领域,而由于军事设施与汽车在GNSS设备条件上的差异,这些成果往往不能直接应用到智能网联汽车上。而目前少部分聚焦于智能网联汽车GNSS位置欺骗研究的工作,往往仅关心GNSS位置欺骗攻击的检测方法,而忽略了攻击发生后可能会对汽车带来的功能安全危害,以及在攻击已经无法挽回的情况下如何采取应急策略来最小化损失。本文针对上述背景,提出了一种可以用于智能网联汽车GNSS位置欺骗攻击的检测方法,并在此基础上,提出相应的功能安全联动预警与应急策略,构建一个完整的“检测-预警-应急”系统。

\subsection{本文主要工作}
本论文分为六章,内容分别如下:

第一章为引言,主要介绍本论文的研究背景、研究意义、主要工作以及论文的组织结构。

第二章为相关技术简介,主要介绍与本论文工作相关的基础技术细节,包括GNSS原理概述,LSTM原理概述,汽车功能安全以及国内外对本文工作的研究现状。

第三章为基于LSTM进行汽车位置预测的GNSS位置攻击欺骗检测模型,介绍了如何基于LSTM构建一个可用于智能网联汽车的GNSS位置欺骗攻击检测算法。

第四章为GNSS位置欺骗攻击功能安全应急策略,主要介绍应对GNSS位置欺骗攻击的功能安全应急策略,以及如何将检测算法与应急策略进行联动。

第五章为实验与结果分析,主要介绍基于CARLA模拟器的仿真实验细节,以及具体的实验结果与分析。

第六章为结论与展望,主要是简要总结本文工作,并对进一步的研究工作提出展望。

\subsection{本文贡献}
